\section{Introduction}

The potential of social networks have not been leveraged effectively in the context of science, the
scientific community and scientific collaboration. This proposal seeks to bring the true utility of social
networks to science, particularly with respect to scientific discovery, sharing and collaboration. The
Science Network will bring together these aspects, along with a practical platform for performing
experiments, peer review and vetting of scientific discovery.
Social networking is a phenomenon that has revolutionized the way in which the world communicates.
Applications like Facebook [1], Twitter [2], and del.icio.us [3] have become so popular that they have
profoundly influenced the way people conduct their daily lives. People are able to represent themselves
publicly through the web and connect to others by sharing content ranging from simple comments to
other more complex media such as video streams. Often, an important and serendipitous benefit of social
networks is the ability to create concrete models that enable knowledge discovery possibilities about the
nature of how people interact with each other.
Intuitively, scientific communities should find utility in social networks [4]
. Canady [5
he
value of social networks for scientists lies in faster access to information relevant to their research and the
communities that are made more available by new tools. These networks provide a forum for scientists to
converse and engage each other on current scientific activities. Connections can be conveniently made
for people in a more expedient manner online, rather than rely on traditional mechanisms that are
hampered by geospatial (e.g. face-to-face meetings) and temporal (e.g. telecommunications) constraints.
For example, climate modelers that participate and contribute to the Earth System Grid Federation
(ESGF) data portal system may want to share observations about particular simulations and experiments.
Rather than enforce teleconferences or “all-hands” meetups, users of ESGF can contribute these scientific
activities at their leisure. This type of openness leads to faster, more collaborative efforts that can be used
to optimize the scientific discovery process.
] states that t
The concept of a social network for scientists is hardly novel, as a number of implementation efforts have
been initiated. One of the first applications to take advantage of the growing popularity of social
networks is Mendeley [6], which allows scientists to manage and share publications related to their
research. Users of Mendeley -- typically scientists who have published in their respective fields -- have
access to a number of key social networking technologies (e.g. news feeds, comments, profile pages,
collaborative tagging mechanisms, etc) that they can utilize to make connections with other Mendeley
users. Mendeley then has the computational capacity to make intelligent associations between users of
the system based on their publications and their research interests. Another interesting “scientific” social
network is ourSpaces [7]
. OurSpaces builds on the concept of the Web-based Virtual Research
Environment (VRE) [8] to allow scientists to log their activities online and socially connect those
activities to other scientists. They build upon the Open Provenance Model [
9
] to log these activities so
that users can view historical content of scientific study and utilize principles of the semantic web to
resolve heterogeneities over different activity definitions. Perhaps the most ambitious effort currently in
production is ResearchGate [10], whose user base exceeds 1.4 million scientists from many different
disciplines. Researchers are able to connect to a global scientific community and make their work
visible. This is especially appealing to younger, inexperienced scientists that would like to make relevant
connections to top people in their fields. It harnesses the power of the social graph [11] and presents a
user interface similar to popular Facebook, providing new users a relatively small barrier of entry. Users
receive feeds of scientific activities of their “friends” and are able to post messages and comments about
their activities.
Despite the early successes of the aforementioned social networks, there are plenty of skeptics who
suggest that the concept of the “scientific social network” is doomed to fail [12] [13]. While bringing
several pertinent points to their arguments, these and many other critics fail to realize that their criticism
1
is directed at what we call human-centric social networks. Human-centric social networks, despite
leveraging the available social mechanisms and Web 2.0 technologies that foster human-to-human
collaboration and communication, share a vital, fundamental flaw -- they assume that all information
available on the network is readable and processable only by humans. Therefore, scientists who utilize
these networks are still forced to work in isolated environments. Individually, they discern the content
that they extract from the network and use it in their individualized work. In other words, the majority of
their science is still conducted offline. Conversely, a stronger scientific insight can be derived from the
ability to read and understand the interactions and scientific discoveries online. This will allow machines
to process information about the knowledge that is being generated by and presented to scientists. We
thus present the Science Network (SN).
The idea behind the Science Network is similar to the human-centric social network with one notable
exception - scientific “discoveries” are now the focal point. We say that a scientific discovery is any
advance made in the scientific process. The concept of a discovery-centric social network is derived from
the idea that no discovery is an island -- discoveries are often a product of one or more discoveries that
preceded them. For example, The US Constitution was created by combining ideas from the works,
“Two Treatises for Government” by John Locke, and the “The Spirit of the Laws” by Charles de
Montesque1
. The scientific discovery process works in a similar manner. As a concrete example, a
climate scientist may “discover” a new anomaly in a dataset stored as a netCDF [14] file by using the
subsetting tool OpenDAP [15].
If we enrich the human-centric social network model with the discovery-centric methodology, we add an
extra dimension full of knowledge discovery possibilities. In addition to people being connected,
discoveries can be connected to each other. The relationships between discoveries, in addition to the
relationships between people and their discoveries can provide us with invaluable information about
scientific activities online. We can then use machine processable methods over this information to infer
important aspects about the nature of scientific effort and collaboration
.
In this proposal, we will implement and research methodologies in order to realize the Science Network vision. The proposal is organized in the following manner. In Section 2, we introduce our main use case, the Earth System Grid Federation (ESGF) Peer-to-Peer (P2P), in which we will integrate the Science Network. Section 3 describes the overall Science Network framework. Section 4 explores avenues of research paths that may be taken to build the Science Network. Section 5 presents a rudimentary milestone and deliverables list that we intend to follow throughout the course of the project’s lifespan.
Section 6 gives outlines a management plan for the project. Section 7 demonstrates a success matrix. We list our external references in Section 8.