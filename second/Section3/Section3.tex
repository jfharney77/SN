\section{Science Network Overview}

The Science Network offers an exciting venue for scientists who are seeking newer and more productive
collaborative tools and technologies. In order to actualize the vision of this discovery-centric network,
however, we must address several attendant software engineering challenges. We believe that there are
three components that are of particular interest that add to the novelty of the Science Network: (i) the data
model, (ii) the execution platform, and (iii) the user interface. We detail each component below

\subsection{Data Model}

We believe that the Science Network can be easily modeled and that it can be derived from a combination of two prominent sources: (i) the traditional human-centric social network and (ii) the Open Provenance Model (OPM) [Moreau et al]. Computationally, traditional social networks are typically data models represented as graph data structures.

\subsubsection{The Science Network Model}


\subsubsection{The Science Network Model Example}


\subsection{Execution Platform}


\subsection{User Interface}

\subsubsection{The Human-Centric User Interface}

\subsubsection{The Discovery-Centric User Interface}

